\point
Приемо"=сдаточные испытания проводят с целью контроля соответствия  \dut \  требованиям настоящих~ТУ, установленным для данной категории испытаний.

\point
Приемо"=сдаточные испытания проводит \client \ силами и средствами изготовителя в присутствии представителя ОТК.

\point
Приемо"=сдаточные испытания проводят в объеме и последовательности, указанных в таблице \ref{tab:tab_psi}.

\vspace{-8mm}
\begingroup
	\singlespacing
	\renewcommand{\arraystretch}{1.5}
	\centering
	\setcounter{rowcount}{0}
	\begin{longtabu} {@{}|X[3m]|X[cm]|X[cm]|@{}}
	\captionsetup{labelformat=default}
	\caption{ }\label{tab:tab_psi}\\
	\hline
\multirow{2}{3\tabucolX}[-0.7ex]{\centering Наименование испытания и проверки}	& \multicolumn{2}{p{2\tabucolX}|}{\centering Номер раздела, подраздела, пункта} \\ \cline{2-3}
& технических требований & методов испытаний \\ \hline
	\endfirsthead
%
	\captionsetup{labelformat=continued,skip=3pt}
	\caption[]{ }\\
	\hline
\multirow{2}{3\tabucolX}[-0.7ex]{\centering Наименование испытания и проверки}	& \multicolumn{2}{p{2\tabucolX}|}{\centering Номер раздела, подраздела, пункта} \\ \cline{2-3}
& технических требований & методов испытаний \\ \hline
	\endhead
%
\rownumber \ Проверка качества покрытий	&%
\ref{t_kt_pok}	& \ref{m_kt_pok} \\ \hline
\multirow{2}{*}[-1ex]{\rownumber \ Проверка маркировки}%
& \ref{t_mar_ms} & \ref{m_mar_ms} \\ \cline{2-3}
& \ref{t_mar_mar} & \ref{m_mar_mar} \\ \hline
\rownumber \ Проверка ёмкости &%
\ref{t_capacity} & \ref{m_capacity} \\ \hline
\rownumber \ Проверка номинального напряжения &%
\ref{t_volt} & \ref{m_volt}	\\ \hline
%\rownumber \ Проверка автоматической защиты от короткого замыкания выходных контактов &%
%\ref{t_short} & \ref{m_short} \\ \hline
\rownumber \ Проверка автоматической защиты от глубокого разряда &%
\ref{t_undervolt} & \ref{m_undervolt} \\ \hline
%\rownumber \ Испытания на воздействие вибрации одной частоты &%
%\ref{t_m_sin_1} & \ref{m_m_sin_1} \\ \hline
\rownumber \ Проверка на соответствие требованиям к материалам и комплектующим изделиям	&%
\ref{t_smp}	& \ref{m_smp} \\ \hline
\rownumber \ Проверка комплектности	&%
\ref{t_komp} & \ref{m_komp} \\ \hline
\multirow{2}{*}[-1ex]{\rownumber \ Проверка консервации и упаковки*}%
& \ref{t_up1} & \ref{m_up1} \\ \cline{2-3}
& \ref{t_up2} & \ref{m_up2} \\ \hline
%& \ref{t_up3} & \ref{m_up3} \\ \hline
\multicolumn{3}{|p{170mm}|}{Примечания \newline
1 Последовательность проведения испытаний может быть изменена по согласованию с \client. \newline
2 *Данные проверки проводятся только в случае самостоятельной поставки \dut, либо поставки в качестве ЗИП.} \\ \hline
	\end{longtabu}
\endgroup

\point
Приемо"=сдаточные испытания проводят с применением сплошного контроля.

\point
На приемо"=сдаточные испытания и приемку \client \ извещением предъявляют один или несколько \dut, выдержавших предъявительские испытания, проводимые ОТК в порядке, установленном в Приложении~\ref{prav_pre} настоящих~ТУ.

Количество \dut, предъявляемых одним извещением одновременно, согласовывают с \client.

Предъявление \dut \ производит ОТК извещением, подписанным руководством предприятия (директором или главным инженером) и начальником ОТК. К извещению прилагают документы, подтверждающие соответствие \dut \ требованиям настоящих~ТУ, а также протоколы предъявительских испытаний. 

\point
Результаты приемо"=сдаточных испытаний оформляют протоколом приемо"=сдаточных испытаний.

\subpoint
В случае использования при испытаниях продукции средств автоматизированного контроля с применением вычислительной техники для оформления результатов приемо"=сдаточных испытаний вместо указанного протокола допускается машинная форма документа, удостоверяющего соответствие продукции всем требованиям ТУ, установленным для приемо-сдаточных испытаний, подписанного ОТК и \client. Содержание этого документа устанавливают по согласованию с \client.

\subpoint
По согласованию с \client \ результаты предъявительских и приемо"=сдаточных испытаний могут быть оформлены единым протоколом испытаний. В этом случае в протоколе должны быть предусмотрены отдельные графы для записи результатов предъявительских и приемо"=сдаточных испытаний и заключений по результатам испытаний.

\subpoint
На основании протокола приемо"=сдаточных испытаний \client \ в извещении составляет заключение о соответствии \dut \ требованиям настоящих~ТУ и о принятии, либо о возврате (забраковании) \dut.

\point
При получении положительных результатов приемо"=сдаточных испытаний \client \ в извещении приводит заключение о годности \dut \ и о его дальнейшем использовании (о приемке или о передаче на следующий технологический участок, или о передаче на ответственное хранение~и~т.~п.), а также ставит пломбы и соответствующие клейма на продукции, метод простановки и расположение которых должны соответствовать требованиям настоящих~ТУ. В формуляре (паспорте, этикетке) на принятую продукцию \client \ также дает заключение, свидетельствующее о годности продукции и о ее приемке.

\point
При отсутствии \client \ у изготовителя, ОТК должен направить заказчику один экземпляр извещения с заключением по результатам испытаний \dut, или по согласованию с заказчиком вместо извещения направить заказчику акт о соответствии продукции требованиям настоящих~ТУ или о принятых  \dut.

При необходимости заказчик может направить к изготовителю своих представителей для выборочного повторного проведения испытаний  \dut, принятых ОТК, или для проведения приемо"=сдаточных испытаний последующих \dut.

\point 
\dut, не выдержавший испытания, \client \, с изложением в извещении причин возврата или забракования, немедленно возвращает ОТК для выявления причин несоответствия требованиям настоящих~ТУ на продукцию, проведения мероприятий по их устранению, определения возможности устранения брака (устранения дефектов, или исключения дефектных изделий) и повторного предъявления.

Изготовитель принимает меры по идентификации забракованной продукции и предотвращению ее непреднамеренного использования, или поставки заказчику (потребителю). Порядок распоряжения несоответствующей продукцией устанавливают в документации системы качества при учете требований  \ref{prg1} настоящих~ТУ.

При невозможности (нецелесообразности) устранения дефектов (исключения дефектных изделий) \dut \   окончательно бракуют и изолируют от годных.

Причины несоответствия \dut \ требованиям настоящих~ТУ и принятые по ним изготовителем меры отражают в акте об их исследовании и устранении дефектов и причин их возникновения.

\point
\label{prg}
Возвращенное \client \  \dut \   после устранения дефектов (исключения дефектных изделий), принятия мер по их предупреждению, повторной проверки изготовителем, в том числе ОТК, повторных предъявительских испытаний при их положительных результатах повторно предъявляют \client \ извещением с надписью <<Вторичное>>. К извещению прикладывают акт по исследованию и устранению дефектов.

Вторичное извещение подписывают руководство изготовителя (директор или главный инженер) и начальник ОТК.

Если возвращенный \dut \ не будет повторно предъявляться, то предложение по его использованию, акт по исследованию и устранению дефектов \client \ предъявляют вместе с извещением о предъявлении очередного \dut, или в иные сроки, согласованные с \client.

\point
Повторные испытания проводят в полном объеме приемо"=сдаточных испытаний. 

В технически обоснованных случаях в зависимости от характера дефектов, например, при отказе комплектующего изделия межотраслевого применения (далее КИМП), \client \ может проводить повторные испытания только по тем пунктам программы испытаний (ТУ), по которым выявлены несоответствия продукции установленным требованиям, а также по тем, которые могли способствовать возникновению несоответствий, и по которым испытания при первичном предъявлении не проводились. Указанное правило может применяться в случаях, не снижающих показателей качества принимаемой продукции, если технические обоснования принятых решений документально оформлены. \dut, не выдержавшие повторные испытания, забраковывают и изолируют от годных.

\point
\label{prg1}
Решение об использовании окончательно забракованных \dut \ в каждом конкретном случае принимают заказчик или по его указанию \client \ и изготовитель.
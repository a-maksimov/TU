Испытание на воздействие изменения температуры окружающей среды по \ref{t_k_ckl} настоящих~ТУ проводят следующим образом:
%
\begin{enumerate}
	\item \dut \ помещают в термокамеру;
	\item понижают температуру в камере до \kmmax \ и выдерживают при этой температуре $2$~ч;
	\item рекомендуется устанавливать скорость изменения температуры в камере при охлаждении $1$~\degree $\text{C}/\text{мин}$;
	\item температуру в камере повышают до \kpmax \ и выдерживают \dut \ при этой температуре в течение $2$~ч;
	\item скорость изменения температуры в камере при нагревании рекомендуется устанавливать не более $2$~\degree $\text{C}/\text{мин}$;
	\item после истечения времени выдержки при предельной повышенной температуре окружающей среды цикл испытания повторяют еще дважды.
\end{enumerate}

После окончания последнего цикла испытаний \dut \ извлекают из термокамеры и выдерживают в нормальных условиях $2$~ч, производят внешний осмотр и проверки по \treb, \trebafter \ настоящих~ТУ. 

\dut \  считают выдержавшим испытание, если \dut \ соответствует требованиям \treb, \trebafter \ настоящих~ТУ.
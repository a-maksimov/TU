\point
Периодические испытания проводят с целью:
%
\begin{itemize}
	\item периодического контроля качества \dut;
	\item контроля стабильности технологического процесса в период между предшествующими и очередными испытаниями;
	\item подтверждения возможности продолжения изготовления \dut \ по действующей конструкторской (включая ТУ на изделие), технологической документации и НД и ее приемки.
\end{itemize}

\point
Периодические испытания проводит изготовитель на собственной базе при участии и под контролем \client \ при нем, который дает заключение по результатам периодических испытаний.

Периодические испытания может проводить организация заказчика, если это предусмотрено контрактом, или, с согласия заказчика, испытательная организация промышленности по договору с изготовителем продукции. В этом случае в периодических испытаниях участвуют изготовитель и \client \ при нем.

\point
Испытания проводят в объеме и последовательности, которые приведены в таблице \ref{tab:tab_pi}.

\vspace{-8mm}
\begingroup
	\singlespacing
	\renewcommand{\arraystretch}{1.5}
	\centering
	\setcounter{rowcount}{0}
	\begin{longtabu} {@{}|X[3m]|X[cm]|X[cm]|@{}}
	\captionsetup{labelformat=default}
	\caption{ }\label{tab:tab_pi}\\
	\hline
\multirow{2}{3\tabucolX}[-0.7ex]{\centering Наименование испытания и проверки}	& \multicolumn{2}{p{2\tabucolX}|}{\centering Номер раздела, подраздела, пункта} \\ \cline{2-3}
& технических требований & методов испытаний \\ \hline
	\endfirsthead
%
	\captionsetup{labelformat=continued,skip=3pt}
	\caption[]{ }\\
	\hline
\multirow{2}{3\tabucolX}[-0.7ex]{\centering Наименование испытания и проверки}	& \multicolumn{2}{p{2\tabucolX}|}{\centering Номер раздела, подраздела, пункта} \\ \cline{2-3}
& технических требований & методов испытаний \\ \hline
	\endhead
%
\rownumber \ Проверка массы	&%
\ref{t_kt_m} & \ref{m_kt_m} \\ \hline
%
\rownumber \ Проверка автоматической защиты от короткого замыкания выходных контактов &%
\ref{t_short} & \ref{m_short} \\ \hline
%
\rownumber \ Испытание на устойчивость и прочность к воздействию широкополосной случайной вибрации &%
\ref{t_m_shsv_st} & \ref{m_m_shsv_st} \\ \hline
%
\rownumber \ Испытание на устойчивость при воздействии механических ударов многократного действия &%
\ref{t_m_ydar_mng_st} & \ref{m_m_ydar_mng_st} \\ \hline
%
\rownumber \ Испытание на прочность при воздействии механических ударов многократного действия &%
\ref{t_m_ydar_mng_pr} & \ref{m_m_ydar_mng_pr} \\ \hline
%
%\rownumber \ Испытание на воздействие акустического шума &%
%\ref{t_m_noise}	& \ref{m_m_noise} \\ \hline
%
\rownumber \ Испытание на стойкость при воздействии линейного ускорения	&%
\ref{t_m_yskor} & \ref{m_m_yskor} \\ \hline
%
\rownumber \ Испытание на воздействие повышенной температуры среды	&%
\ref{t_k_p} & \ref{m_k_p}\\ \hline
%
\rownumber \ Испытание на воздействие пониженной температуры среды &%
\ref{t_k_m} &\ref{m_k_m} \\ \hline
%
\rownumber \ Испытание на воздействие повышенной влажности	&%
\ref{t_k_vlg} & \ref{m_k_vlg} \\ \hline
%
\rownumber \ Испытание на воздействие изменения температуры среды &%
\ref{t_k_ckl} & \ref{m_k_ckl} \\ \hline
%
%\rownumber \ Испытания на воздействие атмосферного пониженного давления &%
%\ref{t_k_dav} & \ref{m_k_dav} \\ \hline
\rownumber \ Испытания на воздействие атмосферного пониженного давления при повышенной температуре среды и высоком токе разряда &%
\ref{t_k_dav_t} & \ref{m_k_dav_t} \\ \hline
%
\rownumber \ Испытание на воздействие соляного (морского) тумана &%
\ref{t_k_mist} & \ref{m_k_mist} \\ \hline
%
\rownumber \ Испытание на воздействие атмосферных конденсированных осадков (инея и росы) &%
\ref{t_k_rosa} & \ref{m_k_rosa} \\ \hline
%
\multicolumn{3}{|p{170mm}|}{\hspace{2em} Примечание "--- Последовательность проведения испытаний может быть изменена по согласованию с \client.} \\ \hline
	\end{longtabu}
\endgroup

\point
Испытания проводят не реже одного раза в 2 года на одном \dut.

\point
\dut, для проведения очередных периодических испытаний, отбирают из числа \dut, изготовленных в  контролируемом периоде и выдержавших приемо"=сдаточные испытания. Их отбирает \client \ в присутствии представителя ОТК с оформлением заключения в извещении по форме~1 приложения~Д ГОСТ~РВ~15.307. Отбор оформляют актом по форме~7 приложения~Д ГОСТ~РВ~15.307.

Проведение отдельных видов испытаний, входящих в категорию периодических, на различных экземплярах изделий не допускается.

\point
\label{prx}
Конкретные календарные сроки проведения периодических испытаний устанавливают в графиках, которые составляет изготовитель с участием \client.

В графике указывают место проведения испытаний, сроки проведения испытаний, сроки оформления документации по результатам испытаний и представления акта (отчета) периодических испытаний по форме~9 приложения~Д ГОСТ~РВ~15.307 на утверждение. Сроки испытаний, указанные в графике, должны обеспечивать соблюдение норм периодичности испытаний, установленных в настоящих~ТУ.

График проведения периодических испытаний планируют таким образом, чтобы дата окончания периодических испытаний была не позднее срока действия предыдущих периодических испытаний.

График оформляют в соответствии с порядком, установленным документацией системы качества изготовителя, и утверждают руководство изготовителя и \client.

\point  
Если периодические испытания в целом или отдельные виды из состава периодических испытаний по согласованию с заказчиком будут проводить в испытательных организациях заказчика или промышленности, то графики периодических испытаний утверждают заказчик и изготовитель, или заказчик, проводящая испытания организация промышленности и изготовитель продукции.

\point
При получении положительных результатов периодических испытаний качество продукции контролируемого периода считается подтвержденным по показателям, проверяемым в составе периодических испытаний. Также считается подтвержденной возможность дальнейшего изготовления и приемки продукции (по той же документации, по которой изготовлена продукция, подвергнутая данным периодическим испытаниям) до получения результатов очередных (последующих) периодических испытаний, проведенных с соблюдением установленных в настоящих~ТУ сроков периодичности.

Срок, на который распространяются результаты данных периодических испытаний, указывают в акте (отчете) периодических испытаний.

\point
\label{prx2}  
Результаты периодических испытаний оформляют актом (отчетом) по форме~9 приложения~Д ГОСТ~РВ~15.307 в сроки, определенные графиком в соответствии с~\ref{prx} настоящих~ТУ.

Если изделия испытывают у их изготовителя, акт (отчет) подписывают представители изготовителя, в том числе ОТК, и \client \ при изготовителе. Акт (отчет) утверждают руководство изготовителя (директор или главный инженер) и \client \ при нем.

Если изделия испытывают в организации заказчика, акт (отчет) подписывают представители этой организации, представители изготовителя и \client \ при нем. Акт (отчет) утверждают заказчик (или по его указанию "--- руководитель организации заказчика) и руководство изготовителя.

Если изделия испытывают в сторонней организации промышленности, акт (отчет) подписывают представители этой организации, \client \ при ней (при его наличии), представители изготовителя и \client \ при нем. Акт (отчет) в этом случае утверждают заказчик (или по его указанию "--- \client \ при изготовителе, \client \ в организации, проводившей испытания), руководство изготовителя и организации, проводившей испытания.

К акту (отчету) прикладывают протокол периодических испытаний по форме~8 приложения~Д ГОСТ~РВ~15.307, подписанный лицами, проводившими испытания.

\point  
\label{prx1}
Если \dut \  не выдержали периодических испытаний, то приемку и отгрузку принятой продукции приостанавливают с учетом требований \ref{prp1} настоящих~ТУ до выявления причин возникновения дефектов, их устранения и получения положительных результатов повторных периодических испытаний.

Изготовитель совместно с \client \ при нем анализирует результаты периодических испытаний для выявления причин появления и характера дефектов, составляет акт (отчет) по форме~10 приложения~Д ГОСТ~РВ~15.307, в котором приводит перечень дефектов, обнаруженных при периодических испытаниях, причины их появления и мероприятия по устранению дефектов и (или) причин их появления.

Если в результате анализа установлено, что выявленный дефект не влияет на характеристики изделия, не распространяется на партию контролируемого периода и не требует мероприятий по его устранению, то порядок возобновления приемки и отгрузки определяется совместным решением руководства предприятия и \client.

\point   
Если характер дефектов испытуемого изделия снижает его тактико-технические характеристики, то все принятые и не отгруженные изделия, в которых могут быть дефекты, возвращают предприятию"=изготовителю на доработку (замену), а все принятые и отгруженные за контролируемый период, в которых могут быть дефекты, обнаруженные при испытаниях, должны быть доработаны или заменены годными.

Решение о доработке или замене принимают предприятие-изготовитель и \client \ с участием, при необходимости, предприятия"=разработчика и \client \ на этом предприятии.

\point 
Если для выполнения мероприятий по устранению дефектов и их причин, а также для доработки отгруженной продукции, или замены ее годной требуется решение вышестоящей организации изготовителя и (или) заказчика, то перечень дефектов, обнаруженных при периодических испытаниях, и мероприятий по устранению дефектов и их причин, согласованный, при необходимости, с разработчиком, изготовитель направляет заказчику и своей вышестоящей организации.

\point 
Повторные периодические испытания должны быть проведены в полном объеме периодических испытаний на доработанных (или вновь изготовленных) изделиях после выполнения мероприятий по устранению дефектов. При этом к моменту проведения повторных периодических испытаний вместе с испытываемым изделием должны быть представлены материалы (акт, протоколы испытаний~и~пр.), подтверждающие устранение дефектов, выявленных при периодических испытаниях, и принятие мер по их предупреждению.

В технически обоснованных случаях в зависимости от характера выявленных дефектов по согласованию с \client \ (или заказчиком "--- при проведении повторных испытаний в организации заказчика или сторонней организации промышленности) повторные периодические испытания допускается проводить по тем пунктам программы испытаний, по которым обнаружены несоответствия изделий установленным требованиям, а также по тем пунктам, которые могли способствовать возникновению дефектов, и по которым испытания не проводились.

Повторные испытания проводят по графику, согласованному с \client.

Допускается возобновлять приемку изделий (партий продукции) по получении положительных результатов по тем видам повторных испытаний, на которых были обнаружены несоответствия изделий требованиям настоящих~ТУ при первичных периодических испытаниях, и которые могли способствовать возникновению дефектов, до полного завершения повторных периодических испытаний, если не истек срок действия результатов предыдущих периодических испытаний.

Техническое обоснование принятого решения должно быть документально оформлено.

\point
При получении положительных результатов повторных периодических испытаний приемку и отгрузку \dut \ возобновляют. О возобновлении приемки и отгрузки \dut \ заказчик (либо \client \ согласно  \ref{prp2} настоящих~ТУ) письменно извещает изготовителя. Сторонам, указанным в \ref{prp1} настоящих~ТУ, \client \ сообщает о возобновлении приемки и отгрузки продукции.

Отгрузка ранее принятой продукции, требующей доработки (замены), может быть возобновлена после проведения мероприятий по устранению дефектов и причин, их вызывающих, и приемки \client.

\point
\label{prx3}   
При получении отрицательных результатов повторных периодических испытаний заказчик (или по его поручению \client) и вышестоящая организация изготовителя (если такая существует), либо изготовитель, если это оговорено в контракте на поставку, на основании результатов исследований выявленных дефектов и причин их возникновения принимают решение о целесообразности (возможности) дальнейшего изготовления продукции по действующей конструкторской, технологической и нормативной документации и возобновления ее приемки, а также решение по ранее изготовленной продукции, включая принятую и отгруженную, качество которой не подтверждено периодическими испытаниями. При принятии указанного решения учитывают возможные способы утилизации продукции, необходимость соблюдения охраны окружающей среды и безопасности персонала, ресурсосбережение~и~др.

Одновременно решают вопрос о необходимости выполнения новых работ по доработке технической документации и освоению производства данной продукции с проведением новых квалификационных испытаний (при необходимости), если выявлена невозможность устранения причин производства дефектной продукции изготовителем.

\point  
Результаты повторных периодических испытаний оформляют актом (отчетом) с учетом требований  \ref{prx2} настоящих~ТУ.

\point  
Решение об использовании изделий, подвергнутых периодическим испытаниям, в каждом конкретном случае принимают заказчик (или по его указанию "--- \client) и руководитель изготовителя. Принимаемые решения должны отвечать требованиям законодательства об охране окружающей среды и о безопасности.

\point 
При отсутствии у изготовителя \client \ порядок проведения периодических испытаний изготовителем и отчетности по проведенным испытаниям изделий устанавливают решением заказчика.
\point 
Типовые испытания проводят с целью оценки эффективности предлагающихся изменений в  \dut \  и целесообразности их внесения в конструкцию, технологию или рецептуру изготовления, которые могут повлиять на тактико"=технические характеристики изделия и (или) его эксплуатацию, включая безопасность воздействия на личный состав и окружающую среду.

Испытания проводят на изделиях, в конструкцию, рецептуру или технологию изготовления которых внесены предлагающиеся изменения.

Необходимость проведения типовых испытаний определяют:
%
\begin{enumerate}
	\item разработчик, изготовитель и \client \ при этих предприятиях совместным решением, утвержденным заказчиком;
	\item \client \ при изготовителе по согласованию с ним и, при необходимости, с разработчиком и \client \ при нем совместным решением, утвержденным заказчиком;
	\item заказчик и разработчик "--- совместным решением.
\end{enumerate}

\point 
Типовые испытания проводит изготовитель или по согласованию с заказчиком "--- организация заказчика (сторонняя организация промышленности) с участием \client \ при изготовителе и, при необходимости, с участием разработчика продукции и \client \ при нем. При проведении испытаний в организации заказчика (промышленности), проводящей испытания, в них принимает участие также изготовитель. \client \, участвующие в проведении типовых испытаний, дают заключения по их результатам.

\point 
Типовые испытания проводят по программе и методикам, которые, как правило, должны содержать:
%
\begin{itemize}
	\item состав необходимых испытаний из числа приемо-сдаточных и периодических испытаний;
	\item требования к количеству изделий, необходимому для проведения испытаний (с учетом методов выборочного контроля);
	\item указание об использовании изделий, подвергнутых типовым испытаниям.
\end{itemize}

В программу могут быть включены, при необходимости, специальные испытания (например, сравнительные испытания изделий, изготовленных без учета и с учетом предлагаемых изменений; испытания из состава проводившихся ранее испытаний опытных образцов изделий или изделий, изготовленных при постановке их на производство,~и~др.).

Объем испытаний и контроля, включенных в программу, должен быть достаточным для оценки влияния внесенных изменений на тактико"=технические характеристики изделий, в том числе на их взаимозаменяемость и совместимость, надежность, безопасность, производственную и эксплуатационную технологичность.

\point 
Программу и методики типовых испытании разрабатывает изготовитель изделия. Когда это касается внесения изменений в конструкцию, к разработке программы и методик привлекают разработчика изделий.

Программу утверждают (согласовывают) инстанции, которые должны утверждать в установленном порядке изменение конструкторской или технологической документации на изделие.

\point 
Готовность изделий к типовым испытаниям определяют ОТК изготовителя и \client. Изделия для проведения испытаний в количестве, установленном в программе типовых испытаний, при выборочном контроле отбирают \client \ в присутствии представителя ОТК. Отбор изделий, при необходимости, оформляют актом по форме~7 приложения~Д ГОСТ~РВ~15.307.

\point 
Если эффективность и целесообразность предлагаемых изменений подтверждены результатами типовых испытаний, то эти изменения вносят в конструкторскую (технологическую) документацию на изделие в соответствии с порядком, установленным в НД.

Продукцию, изготовленную после внесения изменений в документацию, испытывают, как указано в разделах \ref{prav_op}, \ref{prav_psi}, \ref{prav_pi} настоящих~ТУ.

\point 
Если эффективность и целесообразность предлагаемых изменений не подтверждены положительными результатами типовых испытаний, то предлагаемые изменения в соответствующую утвержденную и действующую техническую документацию на изделие не вносят и принимают решение по использованию изделий, изготовленных для проведения типовых испытаний (в соответствии с требованиями программы испытаний). При этом учитывают возможные способы утилизации, необходимость ресурсосбережения, охраны окружающей среды и  безопасности персонала.

\point 
Результаты типовых испытаний оформляют актом (отчетом) по форме~11 приложения~Д 
ГОСТ~РВ~15.307 и протоколом испытаний с отражением всех полученных при испытаниях фактических данных.

Акт (отчет) подписывают должностные лица, проводившие испытания, и утверждают: \client \ при изготовителе и руководитель изготовителя или руководитель организации заказчика (сторонней организации промышленности), проводившей испытания, или заказчик.

\point
Результаты типовых испытаний считают положительными, если полученные фактические данные по всем видам проверок, включенных в программу типовых испытаний, свидетельствуют о достижении требуемых значений показателей и характеристик изделия (технологического процесса), оговоренных в программе и методиках, и достаточны для оценки эффективности (целесообразности) внесения изменений в конструкторскую документацию на изделие.
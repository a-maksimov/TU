%Испытания на воздействие атмосферных конденсированных осадков (инея и росы) по \ref{t_k_rosa} настоящих~ТУ проводят в следующем порядке:
%%
%\begin{enumerate}
%	\item перед испытанием производится проверка по \treb \ настоящих~ТУ;
%	\item \dut \ помещают в камеру холода и устанавливают температуру окружающей среды минус $20$~\degree C, выдерживают при этой температуре в течение $2$~ч;
%	\item \dut \ извлекают из камеры, помещают в нормальные климатические условия испытаний; 
%	\item \dut \ выдерживают в нормальных условиях в течение $3$~ч, при этом сразу после помещения \dut \ в нормальные условия и через $30$---$60$~мин производится проверка на соответствие требованиям \treb \ настоящих~ТУ.
%\end{enumerate}
Испытания на воздействие атмосферных конденсированных осадков (инея и росы) по \ref{t_k_rosa} настоящих~ТУ проводят в следующем порядке:
%
\begin{enumerate}
	\item перед испытанием производится проверка по \treb \ настоящих~ТУ;
	\item \dut \ помещают в камеру холода, устанавливают в камере температуру равную температуре точки росы и выдерживают \dut \ при этой температуре в течение $2$~ч;
	\item \dut \ извлекают из камеры, помещают в нормальные климатические условия испытаний; 
	\item \dut \ выдерживают в нормальных условиях в течение $3$~ч, при этом сразу после помещения \dut \ в нормальные условия производится проверка на соответствие требованиям \treb, \trebafter \ настоящих~ТУ.
\end{enumerate}

%Примечание "--- Допускается совмещать данное испытание с испытанием на воздействие пониженной температуры окружающей среды.

После окончания испытания производят проверку \dut \ по \trebafter \ настоящих~ТУ.

\dut \ считается выдержавшим проверку по \ref{t_k_rosa} настоящих~ТУ, если \dut \ соответствует \treb \ настоящих~ТУ в процессе испытания и соответствует \trebafter \ настоящих~ТУ после испытания.
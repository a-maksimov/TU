При проведении испытаний в условиях воздействия климатических и механических факторов устанавливают следующие допустимые отклонения поддержания режимов:
%
\begin{itemize}
	\item повышенных и пониженных температур "--- $\pm 2$~\degree C;
	\item повышенной относительной влажности воздуха "--- $\pm 3$~\%;
	\item пониженного атмосферного давления "--- $\pm 5$~\%; % или $0,133$~кПа (1~мм рт.ст.) в зависимости от того, что больше;
	\item по амплитуде перемещения "--- $\pm 10$~\%;
	\item по частоте вибрации "--- $\pm 0,5$ Гц на частотах менее $25$~Гц и $\pm 2$~\% на частотах $25$~Гц и более;
	\item по амплитуде виброускорения "--- $\pm 20$~\%;
	\item по числу циклов "--- $\pm 5$~\%;
	\item по пиковому ударному ускорению "--- $\pm 15$~\%;
	\item по времени "--- $\pm 10$~\%. 
\end{itemize} 

При невозможности измерения параметров изделия без извлечения из испытательной камеры при различных видах испытаний допускается проводить эти измерения вне камеры.

%Примечание --- При температуре воздуха выше 30~\degree C влажность не должна превышать 70\%.

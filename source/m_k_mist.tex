Испытание на воздействие соляного (морского) тумана по \ref{t_k_mist} настоящих~ТУ проводят для проверки коррозионной стойкости материалов и покрытий, применяемых при изготовлении \dut, следующим образом:
%
\begin{enumerate}
	\item \dut \  помещают в камеру соляного тумана, в которой  устанавливают температуру $35$~\degree C и подвергают воздействию соляного раствора;
	\item \dut \ должен быть размещен так, чтобы в процессе испытания брызги раствора из аэрозольного аппарата или пульверизатора, а также капли конденсата с потолка, стен и других частей оборудования камеры не попадали на \dut;
	\item раствор для создания тумана приготавливают из расчета ($50 \pm 3$)~г хлористого натрия (NaCl по ГОСТ~4233) на $1$~л дистиллированной воды;
	\item раствор распыляют пульверизатором, центрифугой аэрозольного аппарата или другим способом. Создаваемый туман в камере должен обладать дисперсностью $1$---$10$~мкм ($95$~\% капель) и плотностью $2$---$3$~$\text{г/м}^3$;
	\item раствор распыляют в течение $15$~мин через каждые $45$~мин.
\end{enumerate}
	
Общая продолжительность испытания "--- $2$~суток.
	
После окончания испытания \dut \ извлекают из камеры, производят внешний осмотр и проверяют на соответствие \treb, \trebafter \ настоящих~ТУ.

\dut \  считают выдержавшим проверку по \ref{t_k_mist} настоящих~ТУ, если после испытания \dut \  соответствует требованиям \treb, \trebafter \ настоящих~ТУ.

По согласованию с заказчиком допускается испытание аппаратуры заменять испытанием образцов покрытий, материалов, комплектующих изделий, коррозионная стойкость которых неизвестна.
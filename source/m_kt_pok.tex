Проверку \dut \ на соответствие требованиям \ref{t_kt_pok} настоящих~ТУ производят внешним осмотром (внешний вид оценивают визуально при дневном и искусственном освещении), сличением с чертежами на данное изделие. При необходимости производится проверка, предусмотренная стандартом на данный вид покрытия.

Оценку коррозийной стойкости, защитных свойств и механической прочности покрытий производят после климатических и механических воздействий на изделие.

На металлических и неметаллических неорганических покрытиях после всех видов испытаний допускаются следующие виды изменений, если они не влияют на работоспособность и безотказность работы изделия:
%
\begin{itemize}
	\item белый налет в виде пятен на цинковых и кадмиевых покрытиях;
	\item повреждение хроматных пленок не более чем на $10$~\% от общей поверхности;
	\item темные пятна на всех матовых покрытиях, для которых допущена разнотонность;
	\item потемнение серебряных покрытий;
	\item незначительное потускнение для всех блестящих покрытий;
	\item изменение окраски на анодно"=окисных покрытиях с наполнением красителем;
	\item следы коррозии в шлицах и на кромках крепежных деталей при возможности зачистки и последующего нанесения на эти места смазки и лака на все изделия принимаемой партии;
	\item белые точки на анодно-окисных покрытиях в количестве не более $10$~шт. на $1$~$\text{м}^2$ или не более $2$~шт. на деталях, поверхность которых менее $0,1$~$\text{м}^2$ (количество допускаемых точек на поверхностях от $0,99$~до~$0,09$~$\text{м}^2$ рассчитывать по прямо пропорциональной зависимости относительно квадратного метра).
\end{itemize}

\dut \ считается выдержавшим проверку по \ref{t_kt_pok} настоящих~ТУ, если не обнаружено отклонений от требований КД.
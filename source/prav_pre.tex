\fakesection{}
Предъявительские испытания готовой продукции ОТК проводит с целью контроля изделий на соответствие требованиям ТУ и определения их готовности для предъявления \client.

\fakesection{}
Предъявительские испытания проводят в объёме не менее ПСИ.

%\fakesection{}
%Каждое изделие, предъявляемое на испытание, должно быть подвергнуто в процессе изготовления технологической тренировке по инструкции \RN И5 под контролем ОТК и \client \ и производственному контролю цехом"=изготовителем на соответствие требованиям технологической документации.

\fakesection{}
На предъявительские испытания изделия предъявляют предъявительским документом, форма которого установлена изготовителем по согласованию с \client.

%\fakesection{}
%Предъявление изделий ОТК производится производственным мастером по журналу предъявления продукции ОТК, форма которого устанавливается по согласованию с ОТК.

\fakesection{}
Рабочее место и средства, используемые при контроле предъявляемых \dut, должны быть проверены, аттестованы и должны соответствовать требованиям НД и инструкции по регулировке. При невыполнении указанных требований проведение испытаний на данном рабочем месте не допускается, и извещение на испытание отклоняется с указанием конкретных причин.

\fakesection{}
Изделие считают прошедшим предъявительские испытания, если оно прошло испытания с положительным результатом, и результаты испытаний оформлены протоколом (форма~5 приложение~Д ГОСТ~РВ~15.307), а в паспорте на принятое изделие и в извещении дано заключение ОТК о годности изделия.

\fakesection{}
Изделия, принятые ОТК, должны быть опломбированы и иметь соответствующие клейма, метод простановки и расположение которых должны соответствовать комплекту документации согласно \RN.

\fakesection{}
Повторное предъявление изделия после отклонения извещения производится по устранению замечаний с актом об их устранении, подписанным мастером участка и начальником цеха или заместителем начальника цеха по технической части. Предъявление считается первичным, в извещении указывается фактическая дата предъявления изделия.

\fakesection{}
При обнаружении в изделии самоустраняющихся дефектов его возвращают изготовителю для анализа и устранения причин возникновения дефектов. Если анализ не выявляет причину отказа, то это изделие испытаниям и приемке не подлежит.

\fakesection{}
Если в результате анализа установлено, что выявленный дефект не связан с качеством изделия (неправильный режим испытаний, ошибка персонала, проводящего испытания, дефекты технологического и испытательного оборудования, проявившиеся в момент испытаний, и другие ситуации), то решение о продолжении испытания такого изделия принимает начальник ОТК после устранения причин, повлекших данный дефект.

\fakesection{}
Изделие, не выдержавшее испытание, ОТК с изложением в извещении причин забракования и возврата возвращает цеху"=изготовителю для выявления причин несоответствия изделия требованиям ТУ, проведения мероприятий по их устранению, определения возможности исправления брака (устранения или исключения дефектных изделий) и повторного предъявления. При невозможности (нецелесообразности) устранения дефектов (исключения дефектных изделий) изделие окончательно бракуют и изолируют от годных.

\fakesection{}
Возвращенное ОТК изделие после устранения дефектов (исключения дефектных изделий) и повторной проверки цехом"=изготовителем при положительных результатах допускается повторно предъявлять ОТК извещением с надписью <<Вторично>>, подписанным руководством предприятия.

К извещению должен быть приложен акт об анализе, устранении дефектов и перепроверки изделия цехом"=изготовителем и перечень проведенных мероприятий.

Форма акта устанавливается предприятием"=изготовителем.
%Форма акта устанавливается предприятием"=изготовителем и согласовывается с \client.

Если возвращенное изделие не будет повторно предъявляться, то предложение о его использовании, акт об анализе и устранении дефектов и (или) причин их возникновения цех"=изготовитель предъявляет вместе с извещением о предъявлении очередного одноименного изделия или позже в сроки, согласованные с начальником ОТК.

\fakesection{}
Повторные предъявительские испытания проводят в объеме проверок, установленных для предъявительских испытаний. В зависимости от характера дефектов, выявленных при первичных испытаниях, в отдельных технически обоснованных случаях повторные предъявительские испытания могут проводить только в объеме тех проверок, по которым выявлены несоответствия изделий установленным требованиям, которые могли повлиять на возникновение дефектов, и по которым испытания не проводились.

\fakesection{}
Окончательно забракованные по результатам предъявительских испытаний изделия изолируют от годных.

Решение об использовании окончательно забракованных изделий принимают заказчик (или по его поручению \client) и изготовитель.

\fakesection{}
Изделия, прошедшие испытания с положительными результатами, направляются на участок для полного комплектования.

\fakesection{}
Укомплектованное в соответствии с требованиями КД изделие вместе с комплектом тары для упаковки изделия начальник участка предъявляет ОТК для приемки по журналу предъявления продукции ОТК, форма которого устанавливается по согласованию с ОТК.

\fakesection{}
ОТК проводит контроль комплектности и осмотр изделия и тары по \ref{m_komp}, \ref{m_up} настоящих~ТУ. При обнаружении некомплектности изделие возвращают изготовителю, а в журнале предъявления ОТК дает заключение о причине возврата.

\fakesection{}
Изделие может быть предъявлено вновь с записью в журнале предъявления <<Вторично>> при наличии документов, подтверждающих устранение обнаруженных несоответствий и перепроверку данного комплекта изделия.

\fakesection{}
Принятые ОТК изделия предъявляются \client \ в соответствии с \ref{prav_psi} настоящих~ТУ.
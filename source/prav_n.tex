Соответствие \dut \ требованиям настоящих ТУ по надежности оценивают экспериментальным методом в составе изделия по результатам его эксплуатационных испытаний (подконтрольной эксплуатации) в заданных условиях его эксплуатации. 

В процессе подконтрольной эксплуатации изделия фиксируется первичная статистическая информация о надежности \dut \ (наработка, отказы и причина их возникновения, время восстановления \dut \ в работоспособное состояние, мероприятия по установлению причин отказов). 

По результатам анализа, оценки статистической информации о надежности \dut \ принимается согласованное с \client \ решение о соответствии (несоответствии) \dut \ требованиям ТУ.

При несоответствии \dut \ требованиям ТУ или неэффективности принятых предприятием"=разработчиком (предприятием"=изготовителем) мер по повышению надежности \dut \, предприятие"=разработчик совместно с предприятием"=изготовителем проводит анализ причин несоответствия и разрабатывает план мероприятий по устранению причин отказов и повышению надёжности.

В отдельных случаях, по решению \client, на отдельно выделенных \dut \ могут проводиться специальные испытания на надежность, для определения соответствия \dut \ требованиям ТУ по наработке на отказ. В этом случае по указанию \client \ предприятие"=разработчик (предприятие"=изготовитель) разрабатывает методику и программу проведения испытаний и согласовывает с \client.

%%%%%%%%%%%%%%%%%%%%%%%%%%%%%%%%%%%%%%%%%%%%%%%%%%%%%%%%%%%%%%%%%%%%%%%%%%%%%%%%%%%%%%%%%%%%%%%%%%

%\point
%Испытания на надежность проводят с целью определения количественных показателей надежности изделий и соответствия их~\ref{t_n} настоящих~ТУ.
%
%\point
%Испытания на надежность проводят один раз в четыре года в сроки, определенные совместным решением руководства предприятия"=изготовителя и \client.
%
%Испытаниям подвергается изделие, прошедшее приемо"=сдаточные испытания или приемо"=сдаточные и периодические испытания. В последнем случае наработка изделия за время периодических испытаний засчитывается в продолжительность испытаний на надежность, а отказы учитываются при подсчете результатов этих испытаний.
%
%\point
%Изделие для проведения испытаний на надежность отбирает \client \ в присутствии представителя ОТК предприятия"=изготовителя из числа изделий, изготовленных в контролируемом периоде и выдержавших приемо"=сдаточные испытания с оформлением заключения в извещении по форме~1 приложения~Д ГОСТ~РВ~15.307.
%
%Отбор изделий оформляют актом по форме~7 приложения~Д~ГОСТ~РВ~15.307.
%
%\point
%Испытания проводятся предприятием"=изготовителем под контролем \client. Методика испытаний на надежность приведена в~\ref{m_n} настоящих~ТУ.
%
%Результаты испытаний оформляются актом. В акте указывается срок, на который распространяются результаты испытаний на надежность. Акт подписывается начальником отдела надежности, ОТК и \client.
%
%Акт утверждается руководством предприятия"=изготовителя и \client.
%
%\point
%В случае несоответствия полученных при испытаниях количественных показателей надежности требованиям технических условий на изделие, отгрузка изделий приостанавливается и создается комиссия из представителей предприятия"=разработчика, предприятия"=изготовителя и \client \ при нем для анализа причин отказов, имевшихся при испытаниях, выработки мероприятий по повышению надежности изделия и определения характера и объема необходимых доработок.
%
%Решение о порядке дальнейшего выпуска изделий принимается лицами, утвердившими ТУ.
%
%\point  Решение об использовании изделий, прошедших испытания на надежность, в каждом конкретном случае принимает \client \ и руководство предприятия"=изготовителя.


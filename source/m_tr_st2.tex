Испытания на устойчивость \dut \ к воздействию импульсного высокочастотного сигнала по \ref{t_tr_st2}  настоящих~ТУ проводят следующим образом. 


В испытательной схеме (рисунок \ref{fig:ISOL}), без подключения \dut. Вместо испытуемого объекта вставить переход тип IV Р - IV Р.

На выходе источника импульсного высокочастотного сигнала устанавливают параметры, указанные в~\ref{t_tr_st2} настоящих~ТУ. 

Включить источник высокочастотного сигнала. Получить требуемое значение мощности на измерителе мощности N1912A и форму импульса сигнала на осциллографе GDS-2062. Выключить источник высокочастотного сигнала.

Собрать схему стенда согласно рисуноку \ref{fig:ISOL} с включением Ф3/1В.

Установить на источнике питания Б5-66М напряжение +5 В, ток 0,3 мА.
Установить на источнике питания Б5-67М напряжение "минус" 70 В, ток 0,04 мА.

Перевести тумблеры на устройстве подачи напряжения смещения в режим соответствующий фазовому состоянию 0° (1.1.6 настоящих~ТУ).
Включить источник высокочастотного сигнала. Длительность воздействия 1,0 мин, в течение которой контролируются форма импульса сигнала на осциллографе GDS-2062 и изменения значения мощности на измерителе мощности N1912A.
Выключить источник высокочастотного сигнала.

Перевести тумблеры на устройстве подачи напряжения смещения в режим соответствующий фазовому состоянию 315° (1.1.6 настоящих~ТУ).
Включить источник высокочастотного сигнала. Длительность воздействия 1,0 мин, в течение которой контролируются форма импульса сигнала на осциллографе GDS-2062 и изменения значения мощности на измерителе мощности N1912A.
Выключить источник высокочастотного сигнала.


\begin{figure}[ph]
\begin{center}
\includegraphics[scale=0.8]{isol.pdf}
\end{center}

\begin{center}
\caption{Схема проверки устойчивости  \dut \ к воздействию импульсного высокочастотного сигнала }
\label{fig:ISOL}
\end{center}
\end{figure}


\dut \ считают выдержавшим испытания по \ref{t_tr_st2} настоящих~ТУ, если в процессе испытания форма импульса и значение мощности на измерителе мощности остаются, в пределах погрешности приборов, неизменными в течение 1,0 мин действия сигнала передатчика.
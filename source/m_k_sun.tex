Испытание на воздействие атмосферного пониженного давления по \ref{t_k_dav} настоящих~ТУ проводят следующим образом.

Перед испытанием производятся проверки \dut по \ref{t_VSWR}, \ref{t_oclab}, \ref{t_zatyx},  \ref{t_phase}, \ref{t_kt_TY}, \ref{t_kt_pok}, \ref{t_mar_mar} настоящих~ТУ.

\dut помещают в камеру и собирают схему для проверки по \ref{t_pover} настоящих~ТУ. 

Средства измерения подготавливают к работе согласно прилагаемым к ним инструкциям по эксплуатации.

Аппаратуру включают, устанавливают на осциллографе форму и размер контролируемого импульса и начинают понижать в камере давление до значения, установленного в \ref{t_k_dav} настоящих~ТУ.  

\dut выдерживают при заданном давлении в течение 1 ч, постоянно контролируя форму импульса, затем аппаратуру выключают и давление повышают до нормального. 

В течение всего времени изменения давления до заданного не должно быть резких искажений формы контролируемого импульса, характерного для высокочастотного пробоя.

\dut извлекают из камеры и осматривают.

\dut считается выдержавшей проверку по \ref{t_k_dav} настоящих~ТУ, если в процессе проведения испытания он соответствует \ref{t_pover} настоящих~ТУ и после испытания \dut соответствует \ref{t_VSWR}, \ref{t_oclab}, \ref{t_zatyx},  \ref{t_phase}, \ref{t_kt_TY}, \ref{t_kt_pok}, \ref{t_mar_mar} настоящих~ТУ.
\point
\dut \ до его отгрузки или передачи заказчику подлежит испытаниям и приемке с целью удостоверения  в его годности к использованию в соответствии с требованиями, установленными настоящими ТУ.

\point  
Для контроля качества и приемки \dut \ подвергают следующим категориям испытаний\footnote{%
Приемо"=сдаточные, периодические и типовые испытания проводятся в случае закрепления за предприятием"=изготовителем \client \ или направления на предприятие"=изготовитель представителя МО РФ для осуществления контроля качества и приемки продукции.}:
%
\begin{itemize}
	\item \hyperref[prav_kv]{квалификационным};
	\item \hyperref[prav_psi]{приемо"=сдаточным};
	\item \hyperref[prav_pi]{периодическим}; 
	\item \hyperref[prav_tip]{типовым}.
%	\item \hyperref[prav_n]{на надежность}.
\end{itemize}

\point 
Применяемые средства испытаний, измерений и  контроля, а также методики измерений и контроля должны соответствовать требованиям действующей НД по метрологическому обеспечению.

Не допускается применять средства испытаний, измерений и контроля, не прошедшие метрологическую аттестацию (поверку) в установленном порядке.

Перечень средств измерений, применяемых при испытаниях, приведен в~Приложении~\ref{equip} настоящих~ТУ.

\point 
\dut, предъявляемый на испытания, должен быть укомплектован в соответствии с требованиями настоящих~ТУ (при типовых испытаниях "--- с требованиями программ и методик испытаний). Используемые для комплектации покупные и получаемые по кооперации изделия должны пройти входной контроль, осуществляемый по ГОСТ~24297 и соответствующим инструкциям по входному контролю (при их наличии).

\point 
В процессе испытаний не допускается подстраивать (регулировать) \dut.

\point
Результаты испытаний \dut \ считают положительными, а \dut \ выдержавшим испытания, если он испытан в полном объеме и последовательности, которые установлены в настоящих~ТУ для данной категории испытаний, и соответствует всем требованиям, указанным в настоящих~ТУ и проверяемым при этих испытаниях.

\point
Результаты испытаний \dut \  считают отрицательными, а \dut \ не выдержавшим испытаний, если установлено несоответствие  \dut \ хотя бы одному требованию настоящих~ТУ для данной категории испытаний.

\point 
В технически обоснованных случаях по согласованию с заказчиком каждую категорию контрольных испытаний можно проводить в несколько этапов. При этом \dut \ допускается предъявлять по отдельным извещениям для каждого этапа испытаний.

\point
Результаты испытаний \dut \ по каждой категории испытаний должны быть документально оформлены, в том числе и результаты поэтапных испытаний.

\point
Дефекты \dut, выявленные в ходе приемо"=сдаточных, периодических и типовых испытаний, а также обнаруженные \client \ при контроле качества  \dut, сборочных единиц, деталей и операций технологического процесса на любом этапе производства, должны быть проанализированы изготовителем с участием \client. Результаты анализа и мероприятия по устранению и предупреждению выявленных недостатков должны быть оформлены документально и согласованы с \client.

\point
При проведении испытаний и приемки у изготовителя продукции материально"=техническое и метрологическое обеспечение (необходимая документация, справочные материалы, рабочие места, средства испытаний, измерений и контроля, расходные материалы и др.), а также выделение обслуживающего персонала, охраны, транспортных средств, средств связи и прочего осуществляет изготовитель.

При проведении испытаний в организациях заказчика или промышленности (полигон, специализированный институт, испытательный центр и т. д.) материально"=техническое, метрологическое и бытовое обеспечение, выделение обслуживающего персонала, охраны, транспортных средств, средств связи и прочего осуществляют указанные организации и изготовитель согласно заключенным контрактам (согласованным решениям).

\point
Изготовитель и проводящие испытания организации обеспечивают своевременное проведение испытаний и строгое соблюдение правил техники безопасности при испытаниях.

Предъявление \dut \ на испытания и приемку \client \ должно осуществляться ритмично с установлением, при необходимости, календарных сроков предъявления.

\point
Контроль качества и приемку \dut \ \client \ проводит в присутствии представителя отдела технического контроля (далее ОТК) силами и средствами изготовителя, в объемах и последовательности, установленных в настоящих~ТУ. 

\point
Испытания и приемку \dut \ проводят в один общий этап, содержащий в том числе приемо"=сдаточные испытания. \dut \ предъявляют одним общим предъявительским документом "--- на приемо"=сдаточные испытания и приемку (в том числе по каждому этапу).

\point
Предъявление \dut \ на испытания и приемку осуществляют поштучно, либо партиями, либо совокупностью нескольких изделий или партий продукции, что отражают в предъявительском документе.

Приемку полностью укомплектованной готовой продукции осуществляют по получении извещения.

Допускается предъявлять \dut \ на приемку по журналу или в иной форме по согласованию с \client.

\point
Основанием для принятия решения о приемке \dut \ являются положительные результаты его приемо"=сдаточных испытаний, проведенных в соответствии с действующей технической документацией, а также положительные результаты предыдущих периодических испытаний при условии, что установленные в настоящих~ТУ сроки подтверждения этими испытаниями возможности изготовления и приемки \dut \ не истекли.

Приемке \dut, выпуск которых изготовителем начат впервые, должны предшествовать квалификационные испытания, проводимые в соответствии с ГОСТ~РВ~15.301. Результаты квалификационных испытаний являются основанием для решения вопросов приемки продукции в период после их проведения вплоть до получения результатов очередных (первых) периодических испытаний.

Приемке \dut, выпуск которых изготовителем возобновлен после перерыва на время, превышающее срок периодичности, установленный для периодических испытаний данной продукции, должны предшествовать периодические испытания.

\point
\label{prp}
Испытания и приемку \dut \ приостанавливают в следующих случаях:
%
\begin{enumerate}
	\item если \dut, предъявлявшийся дважды на приемку, не выдержал приемо"=сдаточных испытаний оба раза;
	\item если экземпляры \dut, последовательно один за другим первично предъявлявшиеся на приемо"=сдаточные испытания, не выдержали их и были окончательно забракованы (без права их повторного предъявления на приемку) по результатам каждых из двух последовательно проведенных первичных приемо"=сдаточных испытаний;
	\item \label{itm:prp_3} если \dut \ не выдержал периодических испытаний;
	\item если при контроле качества изготовления \dut \ выявлены дефекты, причиной которых является несоответствие технологических процессов установленным требованиям (в том числе обнаружено несоответствие средств испытаний, измерений и контроля установленным требованиям);
	\item \label{itm:prp_5} если в процессе эксплуатации \dut \ обнаружены дефекты и конструктивные недоработки, вызывающие отказ изделий, и установлено, что эти дефекты и конструктивные недоработки имеются также в \dut \, находящихся в производстве;
	\item \label{itm:prp_6}если не выполняются в срок принятые решения по обеспечению качества продукции;
	\item \label{itm:prp_7} если продолжается изготовление \dut \ без внесения в техническую документацию в установленный срок изменений, предусмотренных контрактом или другими двухсторонними документами;
	\item \label{itm:prp_8} если в процессе изготовления \dut \ обнаружится их несоответствие обязательным требованиям государственных и отраслевых стандартов и условиям контракта на поставку.
\end{enumerate}

\subpoint
В случае приостановки приемки \dut \ разрешается после получения результатов исследований обнаруженных отклонений от НД в продукции или технологическом процессе и фиксирования причин их возникновения продолжать изготовление и приемку деталей и сборочных единиц собственного производства, не подлежащих самостоятельной поставке и не являющихся причиной приостановки приемки продукции.
\subpoint
В случаях приостановки приемки, указанных в пунктах <<\ref{itm:prp_3}>>, <<\ref{itm:prp_5}>>, <<\ref{itm:prp_6}>>, <<\ref{itm:prp_7}>>, <<\ref{itm:prp_8}>> \ref{prp} настоящих~ТУ, приостанавливают также отгрузку принятой продукции

\point
\label{prp1}
В случае приостановки приемки и отгрузки продукции \client \ письменно уведомляет об этом заказчика, изготовителя и потребителя, заключившего контракт с изготовителем на поставку продукции, и \client \ при нем.

\point
\label{prp2}
Решение о возобновлении приемки и отгрузки продукции принимает заказчик после проведения изготовителем согласованных с \client \ мероприятий по устранению причин, вызвавших приостановку приемки и отгрузки продукции, и оформления соответствующего документа, согласованного с \client.

Допускается решение о возобновлении приемки и отгрузки продукции принимать на уровне руководителя изготовителя и \client \ по согласованию с заказчиком. В этом случае причины приостановки приемки и отгрузки и принятые изготовителем меры по устранению дефектов \client \ сообщает в установленном порядке заказчику.

Если приемка продукции была приостановлена вследствие отрицательных результатов периодических или других испытаний, выделенных из периодических испытаний в отдельную категорию, то решение о возможности возобновления приемки принимают в соответствии с \ref{prx1}--\ref{prx3} настоящих~ТУ после выявления причин возникновения дефектов, их устранения и получения положительных результатов повторно проведенных периодических испытаний (либо, в обоснованных случаях, тех видов испытаний, входящих в категорию периодических испытаний, при проведении которых были обнаружены дефекты, или которые могли повлиять на их возникновение, при условии, что не истекли сроки действия результатов предшествующих периодических испытаний).

\point
Принятыми считают \dut, которые: выдержали приемо"=сдаточные испытания при соблюдении действия на них положительных результатов периодических испытаний; промаркированы; укомплектованы; подвергнуты консервации и упакованы в соответствии с требованиями настоящих~ТУ и условиями контрактов на поставку продукции; опломбированы ОТК и \client \ и на которые оформлены документы, удостоверяющие приемку продукции \client.

\point
Принятая продукция подлежит отгрузке или передаче изготовителю на ответственное хранение. Изготовитель должен обеспечить сохранность качества и комплектности продукции после ее приемки вплоть до доставки к месту назначения, если иное не оговорено условиями контракта на поставку.

\point
Если поставку изделий осуществляют без приемки \client, то с письменного разрешения заказчика или в соответствии с условиями контракта на поставку приемо"=сдаточные испытания и приемку осуществляет ОТК изготовителя. В этом случае предъявительские испытания в порядке, установленном в~\ref{prav_pre}, проводит служба изготовителя, предъявляющая \dut \ ОТК для проведения им приемо"=сдаточных испытаний.

\point
Если на предъявленную на контроль качества и приемку продукцию неправильно оформлена документация, не подготовлены рабочие места или средства испытаний, измерений и контроля, не выполнены согласованные \client \ мероприятия (решения) и не приняты меры по устранению недостатков, обнаруженных в процессе летучего контроля \client, то предъявленные \dut \ могут быть отклонены от приемки с указанием причин отклонения на предъявительском документе.

Порядок повторного предъявления продукции должен соответствовать требованиям \ref{prg} настоящих~ТУ, если отклоненную по требованиям этого пункта продукцию \client \ не квалифицирует как первично предъявляемую в соответствии с требованиями документации системы качества изготовителя, согласованной с \client.

\point
На принятый \dut, подлежащий поставке, \client \ выдает изготовителю удостоверения, являющиеся основанием для предъявления счетов к оплате, и контролирует правильность оформления платежных документов. Удостоверения по установленной форме с двумя заверенными \client \ копиями оформляют в течение $24$~ч с момента приемки \dut.
Подтверждение требований \ref{t_n_fail} проводится по результатам контрольных испытаний на надежность в сроки и в объеме, согласованные с \client \ на предприятии"=изготовителе.

Количество изделий, отобранных на испытания должно быть не менее двух и согласовывается с представителем заказчика.

План испытаний, режимы, условия и правила оценки результатов испытаний устанавливаются в соответствии с требованиями ГОСТ~РВ~20.57.304:

\begin{itemize}
	\item шифр плана контроля В"~6 (последовательный метод);
	\item $\alpha = \beta = 0,2$ (риск поставщика и риск заказчика);
	\item $T0/T1 = 2$ (T0 "--- приемочное значение наработки на отказ, T1 "--- браковочное значение наработки на отказ).
\end{itemize}

Графическое представление плана испытаний приведено на~\ref{fig:fail}.
%\begin{figure}[!htbp]
%			\centering
%			\includegraphics{fail.gif}
%			\begin{picdescription}
%				\item линия соответствия;
%				\item линия несоответствия;
%				\item\label{d:stend} \hyperref[e:stend]{\stend \ \stendRN};
%			\end{picdescription}
%			\caption{}
%			\label{fig:fail}
%		\end{figure}
1 --- линия соответствия;r --- количество отказов;
2 --- линия несоответствия;Тисп. --- время испытаний.
Последовательность и нормы механических и климатических факторов, воздействующих на изделие в каждом цикле приведены в таблице 23.
Таблица 23
Состав и последовательность испытаний	Время воздействия механических и климатиче-ских факторов в одном цикле, часов
Ударные нагрузки	3 
Вибрационные нагрузки	82
Повышенная влажность	240
Циклическое воздействие температур	240
Нормальные условия	480 - Тув,
 где Тув -продолжительность воздействия ударных и вибрационных нагрузок.
Испытанию на ударные и вибрационные нагрузки не подвергают, если изделие испытывалось на прочность при воздействии ударов и вибрации.

Параметры контролируют на соответствие \treb \ настоящих~ТУ перед началом и после окончания каждого вида механического и климатического воздействия.

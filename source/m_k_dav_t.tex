Испытание на воздействие атмосферного пониженного давления при повышенной температуре окружающей среды и высоком токе разряда по \ref{t_k_dav_t} настоящих~ТУ проводят следующим образом:

\begin{enumerate}
	\item \dut \ помещают в термобарокамеру, температуру в которой доводят до указанного в \ref{t_k_dav_t} значения, и собирают схему для проверки по \treb \ настоящих~ТУ в соответствии с рисунком~\ref{fig:capacity_camera};
	\item средства измерения подготавливают к работе согласно прилагаемым к ним инструкциям по эксплуатации;
	\item давление в камере понижают до значения, указанного в \ref{t_k_dav_t} настоящих~ТУ;
	\item температуру в камере повышают до максимально возможного значения, но не выше \kpshort; 
	\item \dut \ выдерживают при заданном давлении в течение $1$~ч, затем производят проверку по \treb \ настоящих~ТУ при значении сопротивления Стенда нагрузочного \hyperref[e:stend]{\stendRN} $(0,65 \pm 0,05)$~Ом.
\end{enumerate}

\dut \ считается выдержавшим проверку по \ref{t_k_dav_t} настоящих~ТУ, если в процессе проведения испытания он соответствует \treb \ настоящих~ТУ при номинальном значении сопротивления Стенда нагрузочного \hyperref[e:stend]{\stendRN} \ $R_\text{н} = 0,65$~Ом и после испытания \dut \ соответствует \trebafter \ настоящих~ТУ.
Настоящие технические условия распространяются на \ESKDtheTitle \ \RN \ (далее \dut), предназначенный для обеспечения автономным питанием радиоэлектронной аппаратуры, размещаемой в фюзеляжах и подвесных контейнерах летательных аппаратов.

\dut\  должен нормально работать в составе аппаратуры, находящейся в постоянной эксплуатации в условиях по группам 3.1--3.3 ГОСТ~РВ~20.39.304.
%
\begin{itemize}
	\item случайной широкополосной вибрации в диапазоне частот~\msshv, спектральная плотность ускорения $0,020\, g^2/\text{Гц}$;
	\item механических ударов многократного действия с пиковым ускорением \\ до~\mydarmng, длительностью действия ударного ускорения до~$15$~мс;
	\item акустического шума до \mnoise \ в диапазоне частот от $50$~Гц до $10000$~Гц;
	\item рабочей температуры среды от \km~до~\kp, кратковременно до~\kpshort;
	\item повышенной относительной влажности воздуха~\kvlg \ при температуре~$35$~\degree C;
	\item атмосферного пониженного давления не менее~\kdav;
	\item воздействия соляного (морского) тумана;
	\item атмосферных конденсированных осадков (росы).
\end{itemize}

\dut \ должен быть работоспособен после воздействий предельной пониженной температуры среды \kmmax \ и предельной повышенной температуры среды~\kpmax;

Пример записи изделия в других КД и/или при заказе:
%
\begin{table}[h]
	\centering
	\begin{tabular}{|c|c|}
	\hline
Обозначение  & Наименование \T	\\ \hline
\RN	& \ESKDtheTitle \T	\\ \hline
	\end{tabular}
\end{table}

Критерием отказа изделия при испытании является невыполнение требований \treb \ настоящих~ТУ. Порядок учета и анализа отказов при испытаниях "--- в соответствии с требованиями ГОСТ~РВ~20.57.402, ОСТ4.091.267.
Испытания на воздействие ударных и вибрационных нагрузок проводят в соответствии с требованиями п.1.5 настоящих~ТУ.

Испытание на воздействие повышенной влажности проводят следующим образом. Изделие помещают в камеру влажности, температуру в камере повышают до $(35 \pm 2)$~\degree C. Через $2$~ч после достижения заданной температуры относительную влажность воздуха повышают до $(95 \pm 3)$~\% и этот режим поддерживают в камере в течение всего времени испытаний. Ежесуточно изделие включают, и оно работает в течение $8$~ч. Остальное время суток изделие должно находиться в камере в выключенном состоянии.

Испытание на циклическое воздействие температур проводят воздействием пяти температурных циклов, каждый из которых проводят в следующем порядке:
\begin{enumerate}
	\item изделие помещают в камеру, температура в которой минус 55~\degree C и выдерживают в нерабочем состоянии $1$~ч и в рабочем состоянии $1$~ч;
	\item изделие помещают в камеру, температура в которой минус 55~\degree C и выдерживают в нерабочем состоянии $1$~ч и в рабочем состоянии $1$~ч;
	\item затем изделие выключают и помещают в камеру, температура в которой 60~\degree C. Давление в камере снижают до $2,0$~кПа (15~мм~рт.ст.). В этих условиях изделие выдерживают в рабочем состоянии в течение времени, примерно равного 5~\% от времени наработки в основном цикле.
\end{enumerate}

Примечание "--- За время пребывания изделия в условиях воздействия повышенной температуры и пониженного атмосферного давления допускается снижение температуры воздуха в камере из"~за естественного охлаждения после снижения давления.

Испытания в нормальных условиях проводят в течение времени, дополняющего наработку до заданного в одном цикле. При этом в протоколе фиксируют значения температуры и относительной влажности воздуха помещения.

Аналогично проводят 2 и последующие циклы испытаний. По результатам первого, второго и (или) третьего цикла испытаний определяют реализацию процесса отказов, которая представляет собой ступенчатую линию.
Результаты испытаний считают положительными и испытания прекращают, если ступенчатая линия достигает линии~1 (см.~рисунок~\ref{fig:fail}).

Результаты испытаний считают отрицательными и испытания прекращают, если ступенчатая линия достигает линии~2 (см.~рисунок~\ref{fig:fail}).
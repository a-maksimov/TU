Проверку \dut \ на соответствие \ref{t_volt} настоящих~ТУ производят следующим образом:
%
\begin{enumerate}
	\item \dut \ заряжают согласно \ref{m_charge};
	\item \dut \ выдерживают при нормальной температуре окружающей среды не менее $2$~ч;
	\item собирают схему проверки в соответствии с рисунком~\ref{fig:capacity}, не соединяя жгут~\ref{d:cable} cо Стендом нагрузочным~\ref{d:stend};
	\item средства измерения подготавливают к работе согласно прилагаемым к ним инструкциям по эксплуатации;
	\item переключают мультиметр~\ref{d:multimeter} в режим измерения сопротивления;
	\item измеряют сопротивление Стенда нагрузочного~\ref{d:stend};
	\item вычисляют отношение измеренного сопротивления нагрузки~$R$ к номинальному $R_{\text{н}}$ по формуле~\eqref{eq:load}:
		\begin{equation}\label{eq:load}
			\delta R = \frac{R}{R_{\text{н}}},
		\end{equation}
где $R_{\text{н}} = 1,25$~Ом.
	\item соединяют жгут~\ref{d:cable} со Стендом нагрузочным~\ref{d:stend};
	\item переключают мультиметр~\ref{d:multimeter} в режим измерения напряжения;
	\item измеряют напряжение $U_i$ каждые $\Delta t = (60 \pm 5)$~мин в течение~$\delta R\cdot$\work \ с помощью мультиметра~\ref{d:multimeter};
	\item вычисляют среднее значение напряжения \dut \ за время~$\delta R\cdot$\work \ по формуле~\eqref{eq:volt}:
		\begin{equation}\label{eq:volt}
			\overline U = \sum_{i=1}^{n} \frac{U_i}{n},
		\end{equation}
\end{enumerate}

Примечание "--- Допускается совмещать проверку с \ref{m_capacity}.

\dut \ считают выдержавшим проверку по \ref{t_volt} настоящих~ТУ, если рассчитанное во время испытания среднее значение напряжения \dut \ $\overline U$ не ниже номинального, указанного в \ref{t_volt}.
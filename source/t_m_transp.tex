\dut \ должен соответствовать требованиям \treb, \trebafter \ настоящих~ТУ после воздействия механических факторов в условиях транспортирования, приведенным в таблице \ref{tab:tab_transp}.

\begin{table}[h]
	\centering
	\caption{ }\label{tab:tab_transp}
		\begin{tabu} {@{}|X[cm]|X[cm]|>{\setlength{\baselineskip}{0.8\baselineskip}}X[cm]|X[cm]|X[cm]|@{}}
	\hline
Пиковое ударное ускорение $\text{м}/\text{c}^2\, (g)$ & Допустимая длительность действия ударного ускорения, мс & Предпочти\-тельная длительность действия ударного ускорения, мс & Общее число ударов по трем направлениям & Частота повторения, $\text{уд.}/\text{мин}$  \\ \hline
$147\, (15)$ & $5$---$10$ & $6$ & $20000$ & не более 120 \\ \hline
	\end{tabu}
\end{table}

\begin{flushright}
	(Методика \ref{m_m_transp})
\end{flushright}
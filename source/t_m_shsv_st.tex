\dut \ должен соответствовать \treb \ настоящих~ТУ при испытании и \trebafter \ настоящих~ТУ после испытания в трех плоскостях на устойчивость и прочность к воздействию широкополосной случайной вибрации с параметрами, приведенными в таблице~\ref{tab:tab_shsv}.

\begin{table}[!htbp]%
	\centering
	\renewcommand{\arraystretch}{1.5}
	\caption{ }\label{tab:tab_shsv}
	\begin{tabu} {@{}|X[cm]|X[cm]|X[cm]|@{}}
	\hline
%
Диапазон частот & Спектральная плотность ускорения ($W$), $g^2/\text{Гц}$ & Суммарная среднеквадратичная величина ускорения ($\sigma_{\Sigma}$), $g$ \\ \hline
%
$10$---$2000$ & $0,020$ & $10$ \\ \hline
	\end{tabu}
\end{table}

Изменение частоты вибрации должно быть плавным со скоростью $1~\text{окт}/\text{мин}$. \dut \ не должен иметь резонансов на частотах до~$40$~Гц. Допускается наличие резонансных частот элементов конструкции \dut, если эти резонансы не нарушают нормальное функционирование \dut \ и не снижают ее прочность.

Продолжительность испытания по каждой координатной оси "--- $1$~ч.

\begin{flushright}
(Методика \ref{m_m_shsv_st})
\end{flushright}